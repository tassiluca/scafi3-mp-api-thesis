% !TeX root = ../../main.tex
\chapter{Contribution}

In this chapter it is presented the conceptualization of a proposed Scala-based reference architecture for enabling the development of cross-platform, polyglot and distributed libraries and frameworks.

Its discussion is structured as follows.
%
First, the intents and application scenarios motivating the proposed architecture are discussed.
%
Subsequently, the corresponding requirements are formalized.
%
Thereafter, the elements composing the architecture are presented.
%
Finally, the implications of adopting this design are analyzed.

\section{Intents}

The intents of the proposed architecture are to enable the development of distributed Scala libraries and frameworks to be both \textbf{cross-platform}, that is, able to run on multiple platforms and runtimes; and \textbf{polyglot}, that is, designed to be able to interoperate with its public interface from multiple programming languages.
%
Both intents must be achieved while maintaining a unified and unique version of the components implementing the application logic of the software product.

\section{Application scenarios}

\section{Requirements}

\section{Architectural elements}

\subsection{Cross-platform distribution module}

\subsection{Support for a general cross-platform and polyglot serialization binding}

\subsection{Cross-platform and polyglot library abstraction layer}

\section{Consequences}