% !TeX root = ../../main.tex
\chapter{Context and Motivations}

The goal of this chapter is to provide the context and motivations behind the efforts of this thesis.
%
First, Aggregate Computing is introduced, highlighting the importance of portability and language interoperability in this research area.
%
Then, a brief overview of Scafi3, the Scala 3 library for aggregate programming, is provided...

\section{Aggregate computing: a bird's eye view}

In the realm of distributed complex adaptive systems (CAS) Aggregate Computing \todo{cite} is a research paradigm aiming at program such systems through the lens of \textit{macro-programming}, focusing on the collective behavior of the system as a whole rather than on the individual behavior of its components.
%
This paradigm embraces functional programming principles to ensure composability and modularity of programs, allowing developers to build complex systems in a succinct and declarative manner, abstracting away low-level details about communication and coordination among devices and focusing on the desired collective behavior.

The main abstraction, formalized through the \textit{Field Calculus} \todo{cite}, is that of a \textit{computational field}, a distributed data structure mapping each neighbor device to a value.

\subsection{Portability and interoperability in Aggregate Computing}

\section{Scafi3 overview: a Scala 3 library for aggregate programming}

\todo{Why Scala as super language for multiplatform aggregate computing?}
